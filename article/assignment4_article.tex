\documentclass[10pt,a4paper]{article}
\usepackage[utf8]{inputenc}

\usepackage{amsmath}
\usepackage{amsfonts}
\usepackage{amssymb}
\usepackage{graphicx}
\usepackage{listings}
\usepackage[margin=1.0in]{geometry}
\usepackage{caption}
\usepackage{subcaption}
\usepackage{float}
\usepackage[utf8]{inputenc}
\usepackage{refstyle}


\lstset{numbers=left,
	title=\lstname,
	numberstyle=\tiny, 
	breaklines=true,
	tabsize=4,
	language=Python,
	morekeywords={with,super,as},,
	frame=single,
	basicstyle=\footnotesize\tt,
	commentstyle=\color{comment},
	keywordstyle=\color{keyword},
	stringstyle=\color{string},
	backgroundcolor=\color{background},
	showstringspaces=false,
	numbers=left,
	numbersep=5pt,
	literate=
		{æ}{{\ae}}1
		{å}{{\aa}}1
		{ø}{{\o}}1
		{Æ}{{\AE}}1
		{Å}{{\AA}}1
		{Ø}{{\O}}1
	}
\usepackage{setspace}
\doublespacing
\usepackage{bm}
\usepackage{hyperref}

\begin{document}'
\begin{center}
{\LARGE\bf
FYS4150\\
Project 4 - deadline November 15
}
\\
 \includegraphics[scale=0.1]{uio.png}\\
Sander W. Losnedahl\\
University of Oslo, Autumn 2017
 
\end{center}
\newpage

\begin{center}
{\LARGE\bf Introduction}
\end{center}

\newpage

\begin{center}
{\LARGE\bf Method}
\end{center}

\noindent To be able to start programming, one first needs to find the expressions for the partition function $Z$ with its corresponding energy values $E$, the mean magnetic moment $M$, the specific heat $C_V$ and the susceptibility $X$ as functions of the temperature $T$. All of this using the periodic boundary condition.
\\
The partition function $Z$ is giving by: 


$$
Z = \sum^{M}_{i = 1} e^{-\beta * E_i}
$$ 

\noindent where $\beta$ is the inverse temperature given by $\beta = \frac{1}{kT}$ where k is the Boltzmann constant and T is the temperature, so every expression containing $\beta$ is dependent on the temperature $T$. $E_i$ is the energy for different spin settings given by:

$$
E = -J\sum^{N}_{<kl>} s_ks_l
$$

\noindent where J is the coupling constant and N is the total number of spins. $s_k$ and $s_l$ are the spins of two neighbouring objects in a lattice. Since we are working 2x2 lattice, the total number of combinations are given by $2^4 = 16$, considering we are working with the Ising model where the spins can only be $-1$ or $1$.
\\
\begin{figure}[H]
\centering
\includegraphics[width=0.5\textwidth]{22lattice}
\caption{A two by two lattice and how they interact using the Ising model}
\label{fig:22lattice}
\end{figure}
\noindent From \figref{22lattice} we see how the Ising model works on a 2x2 lattice and from this the energy is calculated. One can observe that energy is non-zero on lattices where all the objects have the same spin ($-8J$) and where the two diagonals have opposite spin from each other ($8J$). All other settings have zero energy. Knowing the energy values we can calculate the mean energy $<E>$ with our specific partition function:

$$
<E> = \frac{1}{Z}\sum^{M}_{i = 1} E_i e^{-\beta E_i}
$$
$$
 = \frac{1}{2e^{-8} + 2e^{8} + 12}\sum^{16}_{1} E_i e^{-\beta E_i}
$$
$$
 = \frac{16e^{-8}-16e^{8}}{2e^{-8} + 2e^{8} + 12} = -7.983928
$$

\noindent Now we can simply put the same energy values into our partition function:

$$
Z = e^{-\beta * -8J} + e^{-\beta * -8J} + e^{-\beta * 8J} + e^{-\beta * 8J} + 12*e^0
$$
$$
Z = 2e^{-8J\beta} + 2e^{8J\beta} + 12
$$

\noindent The magnetization is given by:

$$
M = \sum^{N}_{j = 1} s_j
$$

\noindent Unlike in the energy case, the magnetization does not depend the lattice having the same spin or that the diagonals have opposite spin. The only case where the magnetization is zero in a 2x2 lattice is when half of the objects in the lattice opposite spins. Therefore we get the magnetization values:

$$
M_i = [4 + 2 + 2 + 2 + 2 + 0 + 0 + 0 + 0 + 0 + 0 + -2 + -2 + -2 + -2 + -4]
$$

\noindent To calculate the mean magnetic moment or mean magnetization we use the equation below with our calculated magnetic moment and energy values:

$$
|M| = \frac{1}{Z}\sum^{M}_{i}M_ie^{-\beta E_i}
$$
$$
 = \frac{1}{2e^{-8} + 2e^8 + 12}\sum^{16}_{1}M_ie^{-\beta E_i}
$$
$$
 = \frac{8e^{8} + 16}{2e^{-8} + 2e^8 + 12} = 3.994643
$$



\noindent To calculate the the specific heat, one only needs to know the total number of spins of the lattice. In the 2x2 lattice case the total number of spins is 4, and remember also that $\beta J = 1$ and $k = 1$. The specific heat is then:

$$
C_V = \frac{1}{kT^2}(<E^2> - <E>^2)
$$
$$
 = \frac{1}{kT^2}(\frac{128e^{-8} - 128e^8}{2e^{-8} + 2e^8 + 12} - (\frac{16e^{-8} - 16e^8}{2e^{-8} + 2e^8 + 12})^2)
$$
$$
 = 0.128329
$$

\noindent The term $<E^2> - <E>^2$ is also called the variance of the energy and is precisely calculated just as shown above. The same variance calculation can be applied to calculating the susceptibility, but the energy terms have to be substituted for mean magnetization.
Let's calculate the variance separate from the desired equation this time:

$$
\sigma_M^2 = <M^2>-<M>^2 = \frac{1}{Z}\sum^{M}_{i = 1}M_i^2 e^{-\beta E_i} - (\frac{1}{Z}\sum^{M}_{i = 1}M_i e^{-\beta E_i})^2
$$

$$
 = \frac{32}{2e^{-8} + 2e^8 + 12}(e^{8}+1) - (\frac{8e^8 + 16}{2e^{-8} + 2e^8 + 12})^2 = 0.016004
$$

\noindent Remember that the $\beta = 1$ in the 2x2 lattice case, so the susceptibility would be the same as the variance in that case. We would usually have to calculate the susceptibility with the following equation:

$$
X = \frac{1}{k_bT}(<M^2> - <M>^2) = \frac{1}{k_bT}\sigma_M^2 = 0.010853
$$

\noindent To compute the above values, a standard Metropolis algorithm were implemented (kilde). This algorithm takes in a matrix, only ones or negative ones in this case, and then decides to flip or not flip the current value at a random location in the matrix (lattice). The probability of when to flip or not is in this case given by the Boltzmann distribution $e^{-E_i / T}$. If the algorithm decides to flip, the computed energy at that position in the lattice is added to the total energy of the system. The same happens for the magnetization. The initial value of the energy and magnetization is calculated beforehand. What initial matrix to use, that is a randomly generated matrix or a matrix with only ones in it, is determined by the user. After the Metropolis algorithm is finished, the cumulative energy, squared cumulative energy, cumulative magnetization and cumulative magnetization squared is calculated which is needed to later calculate the mean energy, mean magnetization, specific heat and susceptibility. This algorithm is based on random events and so one needs to perform the algorithm many times before the results are stable. Therefore the whole Metropolis algorithm is looped though what is called Monte Carlo cycles. When the number of cycles is increased, the more stable the results will be. This however, takes a lot of processing power when the number of cycles get to around $10^6$.
\\
To be able to run the program, one needs to parallelize the program, and this was done by using MPI with 8 processors. Using MPI made the processing time much smaller, but it still took a long time for large lattice sizes.
\\
Raw data were produced using c++ while plotting and further calculations were done using MatLab. Further calculations entails finding the probability distribution, phase transitions and the thermodynamic limit.  

\newpage

\begin{center}
{\LARGE\bf Results}
\end{center}

\noindent From the general equations in the method, the following analytical and numerical values were found:

\begin{table} [H]
\caption{Analytical and numerical solutions} \label{tab:table1}
\centerline{
\begin{tabular}{|c|c|c|c|}
\hline
Type & Analytical & Numerical & MC cycles needed\\
\hline
$<E>$ & -7.983928 & -7.98437 & 10 000\\
\hline
$<M>$ & 3.994643 & 3.99483 & 10 000\\
\hline
$C_V$ & 0.128329 & 0.124812 & 100 000\\
\hline
$<X>$ & 0.016004 & 0.0153613 & 100 000\\
\hline
\end{tabular}
}
\end{table}

\noindent To get good results for mean energy and mean magnetic moment took very few Monte Carlo, or MC, cycles, only about 10 000. Then the values would be off only by a factor of about $10^{-3}$. The susceptibility and the specific heat took more MC cycles and correct results started showing around $10^{5}$ for susceptibility and specific heat. For $10^4$, both the specific heat and susceptibility gave unstable results.

\begin{figure}[H]
\centerline{
\includegraphics[width=0.7\textwidth]{energyT1random}
\includegraphics[width=0.7\textwidth]{energyT24random}
}
\caption{Energy versus Monte Carlo cycles for T = 1.0 and T = 2.4 with a random initial matrix}
\label{fig:energyrandom}
\end{figure}

\begin{figure}[H]
\centerline{
\includegraphics[width=0.7\textwidth]{energyT1upspin}
\includegraphics[width=0.7\textwidth]{energyT24upspin}
}
\caption{Energy versus Monte Carlo cycles for T = 1.0 and T = 2.4 with initial spin upwards}
\label{fig:energyupspin}
\end{figure}

\begin{figure}[H]
\centerline{
\includegraphics[width=0.7\textwidth]{magnetizationT1random}
\includegraphics[width=0.7\textwidth]{magnetizationT24random}
}
\caption{Magnetization versus Monte Carlo cycles for T = 1.0 and T = 2.4 with a random initial matrix}
\label{fig:magneticrandom}
\end{figure}

\begin{figure}[H]
\centerline{
\includegraphics[width=0.7\textwidth]{absmagnetizationT1random}
\includegraphics[width=0.7\textwidth]{absmagnetizationT24random}
}
\caption{Absolute value of magnetization versus 10 000 Monte Carlo cycles for T = 1.0 and T = 2.4 with a random initial matrix}
\label{fig:absmagneticrandom}
\end{figure}

\begin{figure}[H]
\centerline{
\includegraphics[width=0.7\textwidth]{magnetizationT1upspin}
\includegraphics[width=0.7\textwidth]{magnetizationT24upspin}
}
\caption{Magnetization versus Monte Carlo cycles for T = 1.0 and T = 2.4 with initial spin upwards}
\label{fig:magneticupspin}
\end{figure}

\begin{figure}[H]
\centerline{
\includegraphics[width=0.7\textwidth]{absmagnetizationT1upspin}
\includegraphics[width=0.7\textwidth]{absmagnetizationT24upspin}
}
\caption{Absolute value of magnetization versus 10 000 Monte Carlo cycles for T = 1.0 and T = 2.4 with initial spin upwards}
\label{fig:absmagneticupspin}
\end{figure}

\noindent We can see from \figref{energyrandom}, \figref{energyupspin}, \figref{magneticrandom} and \figref{magneticupspin} that the mean energy and magnetization quickly approaches it's equilibrium state when the temperature equals $1$. There are of coarse fluctuations after the mean energy and magnetization has reached it's equilibrium state, but the value seems to jump back to to the equilibrium state. Another observation is that the mean energy and magnetization values are more unstable if one starts with an initial matrix where all the spins are oriented upwards. The values become even more unstable when the temperature is increased to 2.4, and also the value of the equilibrium state is changed.

\begin{table} [H]
\caption{EQ values for energy}
\centerline{
\begin{tabular}{|c|c|c|c|}
\hline
Orientation & Temperature & EQ state & max MC cycles needed\\
\hline
Random & 1.0 & -790.581 & 600\\
\hline
Random & 2.4 & -496.494 & 1000\\
\hline
Upwards & 1.0 & -798.96 & 1\\
\hline
Upwards & 2.4 & -493.996 & 300
\label{tab:table2}
\end{tabular}
}
\end{table}

\begin{table}[H]
\caption{EQ values for magnetization} \label{tab:table3}
\centerline{
\begin{tabular}{|c|c|c|c|}
\hline
Orientation & Temperature & EQ state & max MC cycles needed\\
\hline
Random & 1.0 & 398.762 & ???\\
\hline
Random & 2.4 & 165.209 & ???\\
\hline
Upwards & 1.0 & 399.732 & 1\\
\hline
Upwards & 2.4 & 176.087 & ???\\
\hline
\end{tabular}
}
\end{table}

\begin{figure}[H]
\centerline{
\includegraphics[width=0.7\textwidth]{acceptanceMCT1random}
\includegraphics[width=0.7\textwidth]{acceptanceMCT24random}
}
\caption{Acceptance versus Monte Carlo cycles for T = 1.0 and T = 2.4 with a random initial matrix}
\label{fig:acceptancerandom}
\end{figure}

\begin{figure}[H]
\centerline{
\includegraphics[width=0.7\textwidth]{acceptanceMCT1upspin}
\includegraphics[width=0.7\textwidth]{acceptanceMCT24upspin}
}
\caption{Acceptance versus Monte Carlo cycles for T = 1.0 and T = 2.4 with initial spin upwards}
\label{fig:acceptanceupspin}
\end{figure}

\begin{figure}[H]
\centerline{
\includegraphics[width=0.7\textwidth]{acceptanceVStrandom}
\includegraphics[width=0.7\textwidth]{acceptanceVStupspin}
}
\caption{Acceptance versus temperature for a random initial matrix and with initial spin upwards}
\label{fig:acceptancetemp}
\end{figure}

\noindent Acceptance is apparently also dependent on both temperature and the initial state of the spins. From figure \figref{acceptancerandom} one can observe that with higher temperature the number of accepted values increase by a factor of $10$. For $T = 1.0$, one can also observe that the acceptance spikes for few Monte Carlo cycles. 
\\
If the initial spins are all positive like in \figref{acceptanceupspin}, the acceptance becomes more linear. The line for $T = 1.0$ is more squiggly than the straighter line for $T = 2.4$. The ladder does not spike for few Monte Carlo cycles in this case.
\\
From \figref{acceptancetemp}, one can observe that the curves are not linear. The curve based on a randomly generated initial matrix is uneven at low temperatures, but smooths out when $T$ increases. When the initial matrix only consist positive spins, the curve is smooth throughout. These two plots are plotted to a maximum temperature of $3.0$, so one would have an idea of how the acceptance would evolve after $T = 2.4$. 

\begin{figure}
\centerline{
\includegraphics[width=0.7\textwidth]{energyappearanceT24random}
\includegraphics[width=0.7\textwidth]{energyappearanceT24upspin}
}
\caption{Number of appearances per energy value for a random initial matrix and with initial spin upwards when T = 2.4}
\label{fig:energyappearance}
\end{figure}

\noindent With a randomly generated initial matrix, the variance computed in c++ becomes 3226.62. The calculated variance from the energy becomes 3226.9 for $10^6$ MC cycles (calculated in MatLab script "howmanytimes.m"). This gives great confidence in the variance. For an initial matrix where all the spins point upwards, the calculated variance in c++ becomes 3236.05. The variance calculated from energy values in MatLab becomes 3236.3, again showing the stability of the variance for $10^6$ MC cycles. With fewer MC cycles, the two variances calculated does not correlate. Other variances 
\begin{table}[H]
\caption{Variance for different temperatures} \label{tab:table4}
\centerline{
\begin{tabular}{|c|c|}
  Temperature & Variance\\
\hline
  T=1.0 & 10.1076\\
\hline
  T=1.25 & 52.9558\\
\hline
  T=1.50 & 181.261\\
\hline
  T=1.75 & 478.219\\
\hline
  T=2.0 & 1157.08\\
\hline 
  T=2.25 & 3145.36\\
\hline 
  T= 2.50 & 2479.22 \\
\hline 
  T=2.75 & 1690.69\\
\hline
  T=3.0 & 1445.91\\
\hline
\end{tabular}
}
\end{table}

One can observe from \figref{energyappearance} that the curves are slightly skewed towards the higher energy values. The curve is bell shaped and somewhat centred around the equilibrium state for the lattice at temperature $T = 2.4$ which is $-497$ and $-493$. 

\noindent time for 60x60 and 1mill cycles = 3996.57 seconds
\noindent time for 80x80 and 1mill cycles = 7081.15 seconds
\noindent time for 100x100 and 1mill cycles = 11176.2 seconds

\newpage

\begin{center}
{\LARGE\bf Discussion}
\end{center}

\noindent Comparing Table \ref{tab:table2} and to \figref{energyrandom} and \figref{energyupspin} one can observe that the graph approaches the equilibrium value stated in the table. This is very clear for $T = 1.0$, but when $T = 2.4$, the energy seems to be jump around more which is natural due to the increased energy in the system (kilde). Even though the value does not stay at equilibrium, the energy is still said to have reached equilibrium as stated in Table \ref{tab:table2} after a certain number of Monte Carlo cycles.
\\
The magnetization plotted on \figref{magneticrandom} \figref{magneticupspin} show the magnetization without the absolute value and tells us that the magnetization actually jumps from positive to negative magnetization rather quickly and frequent after reaching equilibrium. The reason for this is because the energy values are equal for symmetric magnetization values around the equilibrium state (kilde). This makes it easy for the magnetization to make these kind of leaps. Looking at \figref{absmagneticrandom} and \figref{absmagneticupspin}, the absolute value of the magnetization, we lose the above understanding, but we get the correct values, as calculated in c++, for the equilibrium state.
\\
This statement is only true when $T = 2.4$, because one can observe that when $T = 1.0$ the magnetization does not fluctuate for positive and negative values. It seems as the magnetization reaches it equilibrium state and then shortly jump to some negative value, then right back to the equilibrium value. The reason for this is because the energy in the system is too low for the magnetization to make any great leaps (kilde). From the Boltzmann distribution we can see this mathematically:

$$
w_i = e^{-E_i/T}
$$

\noindent where $w_i$ is the probability, $E_i$ is the energy for and $T$ is the temperature. One can immediately see that when the temperature increase, the probability decreases for low energy values. The probability decreases more for lower values of T, for an example:

$$
E_i = [-8,8] , T = [1.0,2.4]
$$
$$
w_{E_i}(T = 1.0) \approx [2981, 55, 1, 0.02, 0.00034]
$$
$$
w_{E_i}(T = 2.4) \approx [28, 5.3, 1, 0.2, 0.036]
$$

\noindent This means that the likelihood of flipping the orientation of the objects in the lattice should be higher for $T = 2.4$ when the energy starts accumulating to higher values, since the probability should be higher than a random number. The magnetization would then fluctuate more when $T = 2.4$ and this is observed on \figref{magneticrandom}, \figref{absmagneticrandom}, \figref{magneticupspin} and \figref{absmagneticupspin}. The same argument applies for the energy, when the temperature is high, the chance of flipping is higher for $T = 2.4$ compared $T = 1.0$ and one can see this on figure \figref{energyrandom} \figref{energyupspin}.
\\
Following this train of thought, one would expect the rate of acceptance to be greater for higher temperature values and this can be seen by comparing the y-axis on \figref{acceptancerandom} and \figref{acceptanceupspin}. This is confirmed on \figref{acceptancetemp} where the acceptance increases when the temperature increases. When the initial matrix is randomized, the results for low temperatures vary, due to the initial starting condition. This is therefore not the case when the initial matrix only has upwards spin since in this case, the acceptance does not have to stabilize.
\\
Looking at \figref{energyappearance}, one clearly sees that both graphs are skewed towards higher energy values, when one may have expected the curve to center around it's equilibrium state. 





\end{document}